\chapter{半导体的导电性}

\section{载流子的漂移运动和迁移率}

\subsection{欧姆定律}

电阻为$R$的导体两端施加电压$V$,电流为
\begin{equation}
    I=\frac{V}{R}\label{eq:chap-4-ohms-law}
\end{equation}
电阻$R$与导体的长度$l$成正比,与截面积$s$成反比:
\begin{equation}
    R=\rho\frac{l}{s}\label{eq:chap-4-resistance-equation}
\end{equation}
$\rho$为导体的\textbf{电阻率},国际单位$[\rho]=\Omega\cdot\mathrm{m}$,常用单位为$\Omega\cdot\mathrm{cm}$。电阻率的倒数为电导率$\sigma$:
\begin{equation}
    \sigma=\frac{1}{\rho}\label{eq:chap-4-sigma-rho-relation}
\end{equation}
单位为$\mathrm{S}/\mathrm{m}$或$\mathrm{S}/\mathrm{cm}$。

\textbf{电流密度}$J$是通过垂直于电流方向的单位面积截面的电流:
\begin{equation}
    \bm J=\frac{\Delta \bm I}{\Delta s}
\end{equation}
$\bm J$是一个矢量,单位为$\mathrm{A/m^2}$或$\mathrm{A/cm^2}$

一段长$l$,截面$s$,电阻率$\rho$的均匀导体,两端加电压$V$,导体内部电场$\mathscr{E}$大小
\begin{equation}
    \mathscr{E}=\frac{V}{l}\label{eq:chap-4-electrical-field-strength}
\end{equation}
对均匀导体,电流密度
\begin{equation}
    \bm J=\frac{\bm I}{s}\label{eq:chap-4-uniform-conductor-I-density}
\end{equation}
将\autoref{eq:chap-4-electrical-field-strength},\autoref{eq:chap-4-uniform-conductor-I-density}和\autoref{eq:chap-4-resistance-equation}代入\autoref{eq:chap-4-ohms-law},得:
\begin{equation}
    sJ=\frac{\mathscr{E}l}{\D \rho\frac{l}{s}}
\end{equation}
化简得:
\begin{equation}
    J=\sigma\mathscr{E}\label{eq:chap-4-ohms-law-differential-form}
\end{equation}
上式即\textbf{欧姆定律微分形式}。

\subsection{漂移速度和迁移率}

外加电压下,导体电子受电场力作用,沿电场反方向作定向运动形成电流。这种定向运动称为\textbf{漂移运动},定向运动的速度称为\textbf{漂移速度}。用$\Bar{v}_d$表示漂移速度。

导体的任一截面A,设$n$为电子浓度,则单位时间通过的电子数
\begin{equation*}
    nq\Bar{v}_d\mathrm{d}ts
\end{equation*}
则电流为
\begin{equation}
    I=-\frac{nq\Bar{v}_d\mathrm{d}ts}{\mathrm{d}t}=-nq\Bar{v}_ds
\end{equation}
电流密度为
\begin{equation}
    J=\frac{I}{s}=-nq\Bar{v}_d
\end{equation}
恒定电场下,漂移速度与电场强度成正比:
\begin{equation}
    \Bar{v}_d=\mu\mathscr{E}
\end{equation}
$\mu$为电子的\textbf{迁移率},表示单位电场下电子平均漂移速度,单位$\mathrm{m^2/(V\cdot s)}$或$\mathrm{cm^2/(V\cdot s)}$。$\mu$习惯上只取正值:
\begin{equation}
    \mu=\left|\frac{\Bar{v}_d}{\mathscr{E}}\right|
\end{equation}
代入电流密度:
\begin{equation}
    J=nq\mu\mathscr{E}
\end{equation}
比较微分形式欧姆定律\autoref{eq:chap-4-ohms-law-differential-form},得到电导率:
\begin{equation}
    \sigma=nq\mu
\end{equation}
上式即电导率和迁移率的关系。

\subsection{半导体电导率和迁移率}

半导体中同时存在着电子和空穴,记$J_n$为电子电流密度,$J_p$为空穴电流密度,$n,\ p$分别为电子和空穴浓度,则总电流密度:
\begin{equation}
    J=J_n+J_p=\left(nq\mu_n+pq\mu_p\right)\mathscr{E}
\end{equation}
电导率为
\begin{equation}
    \sigma=nq\mu_n+pq\mu_p
\end{equation}
若两种载流子浓度悬殊,迁移率差别不大,则电导率主要取决于多数载流子:
\begin{enumerate}
    \item 对n型半导体,$n\gg p$,空穴对电流的贡献可以忽略,电导率为:
    \begin{equation}
        \sigma=nq\mu_n
    \end{equation}
    \item 对p型半导体,$p\gg n$,电导率为:
    \begin{equation}
        \sigma=pq\mu_p
    \end{equation}
\end{enumerate}
对于本征半导体,有$n=p=n_i$,电导率为
\begin{equation}
    \sigma_i=n_iq(\mu_n+\mu_p)
\end{equation}

\section{载流子的散射}

半导体内的载流子不断进行着\textbf{热运动}。热运动的载流子与晶格原子或电离杂质离子发生碰撞,其速度大小和方向会发生改变,即遭到\textbf{散射}。载流子在两次散射之间自由运动的平均路程称为\textbf{平均自由程},平均时间称为\textbf{平均自由时间}。

\section{迁移率与杂质浓度和温度的关系}

\subsection{平均自由时间与散射概率的关系}

记载流子的平均自由时间为$\tau$。设$0$时刻有$N_0$个电子以速度$v$沿某方向运动,$N(t)$为$t$时刻未散射的电子数。电子受到散射的概率为$P$,$\Delta t$时间内被散射电子数为:
\begin{equation*}
    N(t)P\Delta t
\end{equation*}
故$t$时刻未散射电子数$N(t)$比$t+\Delta t$时刻未散射电子数$N(t+\Delta t)$多$N(t)P\Delta t$:
\begin{equation}
    N(t)-N(t+\Delta t)=N(t)P\Delta t
\end{equation}
当$\Delta t\rightarrow \mathrm{d}t$时:
\begin{equation}
    \frac{\mathrm{d}N}{\mathrm{d}t}=\lim_{\Delta t\rightarrow 0}\frac{N(t)-N(t+\Delta t)}{\Delta t}=-PN(t)
\end{equation}
解这个微分方程,得:
\begin{equation}
    N(t)=N_0\mathrm{e}^{-Pt}
\end{equation}
故$t$到$t+\mathrm{d}t$时刻内被散射的电子数为:
\begin{equation*}
    N_0P\mathrm{e}^{-Pt}\mathrm{d}t
\end{equation*}
在$t$到$t+\mathrm{d}t$时刻内散射的电子的自由时间均为$t$,这些电子自由时间的总和为$N_0P\mathrm{e}^{-Pt}t\mathrm{d}t$。将它为全部时间积分再除以$N_0$即平均自由时间。故平均自由时间有:
\begin{equation}
    \tau=\frac{1}{N_0}\int_0^\infty N_0P\mathrm{e}^{-Pt}t\mathrm{d}t=\int_0^\infty P\mathrm{e}^{-Pt}t\mathrm{d}t=\frac{1}{P}
\end{equation}
即平均自由时间等于散射概率的倒数。

\subsection{电导率、迁移率和平均自由时间的关系}

电子在$0$时刻受到散射后沿$x$方向速度为$v_{x0}$,经$t$时刻再受到散射,此时速度为:
\begin{equation}
    v_x=v_{x0}+at=v_{x0}-\frac{q}{m_n^*}\mathscr{E}t
\end{equation}
故按上节的分析,$N_0$个电子的平均漂移速度$\Bar{v}_x$为:
\begin{align}
    \Bar{v}_x&=\Bar{v}_{x0}-\frac{1}{N_0}\int_0^\infty\frac{q}{m_n^*}\mathscr{E}tN_0P\mathrm{e}^{-Pt}\mathrm{d}t\\
    &=\Bar{v}_{x0}-\int_0^\infty\frac{q}{m_n^*}\mathscr{E}tP\mathrm{e}^{-Pt}\mathrm{d}t
\end{align}
由于$0$时刻速度$\bm v_0$方向随机,故$\bm v_0$在$x$方向上的平均值$\Bar{v}_{x0}=0$。所以:
\begin{equation}
    \Bar{v}_x=-\int_0^\infty\frac{q}{m_n^*}\mathscr{E}tP\mathrm{e}^{-Pt}\mathrm{d}t=-\frac{q\mathscr{E}}{m_n^*}\tau_n
\end{equation}
$\tau_n$为电子平均自由时间。

根据迁移率定义:
\begin{equation}
    \mu=\frac{\left|\Bar{v}_x\right|}{\mathscr{E}}
\end{equation}
得电子迁移率:
\begin{equation}
    \mu_n=\frac{q\tau_n}{m_n^*}
\end{equation}
同理,空穴迁移率:
\begin{equation}
    \mu_p=\frac{q\tau_p}{m_p^*}
\end{equation}
n型材料的电导率:
\begin{equation}
    \sigma_n=nq\mu_n=\frac{nq^2\tau_n}{m_n^*}
\end{equation}
p型材料电导率:
\begin{equation}
    \sigma_p=pq\mu_p=\frac{pq^2\tau_p}{m_p^*}
\end{equation}
混合型材料电导率:
\begin{equation}
    \sigma=nq\mu_n+pq\mu_p=\frac{nq^2\tau_n}{m_n^*}+\frac{pq^2\tau_p}{m_p^*}
\end{equation}

\subsection{电导有效质量}

对于等能面为旋转椭球面的多极值半导体,其晶体沿不同方向有效质量不同。

以Si为例,Si的导带等能面如\autoref{fig:homo-energy-phase}所示。椭圆长轴沿$<1\ 0\ 0>$方向。横向有效质量为$m_t$,纵向有效质量为$m_l$。取$x$轴,$y$轴,$z$轴分别沿$[1\ 0\ 0],\ [0\ 1\ 0],\ [0\ 0\ 1]$方向。设电场强度$\mathscr{E}$\vspace{1ex}
沿$x$轴方向,则电子沿$[1\ 0\ 0]$方向的迁移率$\mu_1=\D \frac{q\tau_n}{m_l}$,其他方向电子迁移率为$\mu_2=\mu_3=\D \frac{q\tau_n}{m_t}$。\vspace{1ex}
设电子浓度为$n$,平均每个能谷单位体积中有$\D \frac{n}{6}$个电子,则电流密度$J_x$为:
\begin{equation}
    J_x=\frac{1}{3}nq(\mu_1+\mu_2+\mu_3)\mathscr{E}
\end{equation}
令:
\begin{equation}
    J_x=nq\mu_c\mathscr{E}
\end{equation}
$\mu_c$为\textbf{电导迁移率}。比较上两式,得:
\begin{equation}
    \mu_c=\frac{1}{3}(\mu_1+\mu_2+\mu_3)
\end{equation}
$\mu_c$可以写成
\begin{equation}
    \mu_c=\frac{q\tau_n}{m_c}
\end{equation}
即:
\begin{align}
    \frac{q\tau_n}{m_c}&=\frac{1}{3}(\mu_1+\mu_2+\mu_3)\\
    &=\frac{1}{3}\left(\frac{q\tau_n}{m_l}+\frac{2q\tau_n}{m_t}\right)
\end{align}
故有:
\begin{equation}
    \frac{1}{m_c}=\frac{1}{3}\left(\frac{1}{m_l}+\frac{2}{m_t}\right)
\end{equation}
$m_c$即为\textbf{电导有效质量}。

\subsection{电阻率}

由\autoref{eq:chap-4-sigma-rho-relation}可知,电阻率是电导率的倒数:
\begin{equation}
    \rho=\frac{1}{\sigma}=\frac{1}{nq\mu_n+pq\mu_p}
\end{equation}
对于n型半导体:
\begin{equation}
    \rho_n=\frac{1}{nq\mu_n}
\end{equation}
p型半导体:
\begin{equation}
    \rho_p=\frac{1}{pq\mu_p}
\end{equation}
本征半导体:
\begin{equation}
    \rho_i=\frac{1}{n_iq(\mu_n+\mu_p)}
\end{equation}














